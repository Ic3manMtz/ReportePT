\section{Caracterización inicial}
\begin{itemize}
	\item Exploración del conjunto de datos, identificación de columnas y filas irrelevantes.
	\item Conteo de registros y dimensiones del dataset.
	\item Inspección de valores únicos y estructura general.
\end{itemize}


\section{Depuración de columnas}
\begin{itemize}
	\item Selección de campos relevantes para el análisis.
	\item Eliminación de columnas redundantes o sin valor analítico.
	\item Conservación de columnas clave: identificador, timestamp, coordenadas, precisión GPS.
\end{itemize}

\section{Depuración de filas}
\begin{itemize}
	\item Eliminación de registros duplicados basados en identificador, timestamp y coordenadas.
	\item Filtrado de registros con baja precisión GPS mayor a 20 metros.
	\item Análisis de frecuencia de aparición de identificadores únicos.
\end{itemize}

\section{Análisis descriptivo}
\begin{itemize}
	\item Distribución de individuos por día mediante histogramas temporales.
	\item Cálculo de frecuencia de aparición y persistencia temporal.
	\item Identificación de individuos con registros en múltiples días.
\end{itemize}

\section{Filtrado por velocidades peatonales}
\begin{itemize}
	 \item Cálculo de velocidades entre puntos consecutivos.
	 \item Creación de histograma de distribución de velocidades.
	 \item Aplicación de filtro peatonal utilizando parámetros estadísticos:
	 \begin{itemize}
	 	\item Media de velocidad peatonal: 1.34 m/s
	 	\item Desviación estándar: 0.37 m/s
	 	\item Rango aceptable: 0.6-2.08 m/s (media \(\pm\) desviaciones estándar)
	 \end{itemize}
	 \item Segmentación de trayectorias para eliminar puntos fuera del rango peatonal
\end{itemize}

\section{Mapa geográfico de trayectorias}
\begin{itemize}
	\item Mapa de distribución de puntos de recorrido por ciudad.
	\item Visualización de trayectorias individuales completas.
	\item Implementación de algoritmo de clustering sobre puntos de velocidad peatonal.
\end{itemize}

\section{Modelado de parámetros de movilidad}
\begin{itemize}
	\item Cálculo de longitud de vuelo entre puntos consecutivos.
	\item Cálculo de tiempos de pausa.
\end{itemize}

\section{Evaluación de calidad}
\begin{itemize}
	\item Desarrollo de algoritmo de puntuación compuesta basado en métricas 
	\item Clasificación de trayectorias en categorías cualitativas.
\end{itemize}