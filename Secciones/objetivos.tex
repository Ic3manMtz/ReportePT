\section{Objetivos}
\noindent El objetivo principal del proyecto es obtener una caracterización estadística de las trayectorias individuales a partir de un conjunto de datos de movilidad. \\

\noindent Los objetivos particulares son:
\begin{itemize}
	\item Caracterizar la base de datos para extraer las trayectorias contenidas.
	\item Aplicar un modelo de inteligencia artificial para identificar y analizar dichas trayectorias.
\end{itemize}

\section{Metodología}
El proceso se divide en las siguientes etapas: \\

{\Large\bfseries 1. Caracterización inicial}
\begin{itemize}
	\item Exploración del conjunto de datos, identificación de columnas y filas irrelevantes.
	\item Conteo de registros y dimensiones de la base de datos.
	\item Inspección de valores únicos y estructura general.
\end{itemize}

\vspace{0.5cm}
{\Large\bfseries 2. Limpieza de datos}
\begin{itemize}
	\item Selección de campos relevantes para el análisis.
	\item Eliminación de columnas redundantes o sin valor analítico.
	\item Conservación de columnas clave: \textit{identificador}, \textit{timestamp}, \textit{coordenadas}, \textit{precisión GPS}.
    \item Eliminación de registros duplicados basados en \textit{identificador}, \textit{timestamp} y \textit{coordenadas}.
	\item Filtrado de registros con baja precisión GPS mayor a 20 metros.
	\item Análisis de frecuencia de aparición de identificadores únicos.
	\item Desarrollo de algoritmo de puntuación compuesta basado en métricas 
	\item Clasificación de trayectorias en categorías cualitativas.
\end{itemize}



\vspace{0.5cm}
{\Large\bfseries 3. Determinación de puntos de recorrido}
\begin{itemize}
	\item Distribución de individuos por día mediante histogramas temporales.
	\item Cálculo de frecuencia de aparición y persistencia temporal.
	\item Identificación de individuos con registros en múltiples días.
	\item Cálculo de velocidades entre puntos consecutivos.
	\item Creación de histograma de distribución de velocidades.
	\item Aplicación de filtro peatonal utilizando parámetros estadísticos:
	\begin{itemize}
		\item Media de velocidad peatonal: 1.34 m/s
		\item Desviación estándar: 0.37 m/s
		\item Rango aceptable: 0.6-2.08 m/s (media \(\pm\) desviaciones estándar)
	\end{itemize}
	\item Segmentación de trayectorias para eliminar puntos fuera del rango peatonal
	\item Mapa de distribución de puntos de recorrido por ciudad.
	\item Visualización de trayectorias individuales completas.
	\item Implementación de algoritmo de \textit{clustering} sobre puntos de velocidad peatonal.
\end{itemize}

\vspace{0.5cm}
{\Large\bfseries 4. Determinación de tiempos de pausa}
\begin{itemize}
	\item Cálculo de tiempos de pausa.
\end{itemize}

\vspace{0.5cm}
{\Large\bfseries 5. Determinación de longitudes de vuelo}
\begin{itemize}
	\item Cálculo de longitud de vuelo entre puntos consecutivos.
\end{itemize}



