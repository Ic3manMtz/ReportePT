\section{Resultados de Clasificación de Calidad de Trayectorias}

El proceso de clasificación de calidad produjo los siguientes resultados a partir del dataset procesado:

\begin{table}[h]
	\centering
	\begin{tabular}{@{}cccc@{}}
		\toprule
		\textbf{Categoría} & \textbf{Cantidad} & \textbf{Porcentaje} & \textbf{Interpretación}                                                                                               \\ \midrule
		REGULAR            & 83,430            & 64.3\%              & \begin{tabular}[c]{@{}c@{}}Trayectorias con información suficiente\\ para análisis básico\end{tabular}                \\
		BUENA              & 40,373            & 31.3\%              & \begin{tabular}[c]{@{}c@{}}Trayectorias con buena cobertura\\ temporal y espacial\end{tabular}                        \\
		MUY BUENA          & 5,863             & 4.5\%               & \begin{tabular}[c]{@{}c@{}}Trayectorias con alta calidad\\ y consistencia\end{tabular}                                \\
		EXCELENTE          & 254               & 0.2\%               & \begin{tabular}[c]{@{}c@{}}Trayectorias ideales con máxima\\ calidad de datos\end{tabular}                            \\
		\textbf{Total}     & \textbf{129,920}  & \textbf{100\%}      & \textbf{\begin{tabular}[c]{@{}c@{}}Individuos con calidad REGULAR\\ o superior (\textgreater 35 puntos)\end{tabular}} \\ \bottomrule
	\end{tabular}
	\caption{Distribución de individuos por categoría de calidad}
	\label{tab:distribucion_calidad}
\end{table}

\paragraph{Análisis de la distribución}
\begin{itemize}
	\item \textbf{Dataset procesado}: Total de 51,077,925 registros después de la limpieza.
	\item \textbf{Umbral de calidad}: Se identificaron 129,920 individuos con calidad REGULAR o superior (puntuación $\geq$35).
	\item \textbf{Representatividad}: Estos individuos representan aproximadamente el 2.15\% del total original (129,920 de 6,022,772 identificadores únicos).
	\item \textbf{Implicación}: El algoritmo de puntuación compuesta logró identificar eficientemente una fracción pequeña pero significativa de individuos con datos de calidad suficiente para análisis de movilidad.
\end{itemize}

\section{Resultados de Extracción de Trayectorias Peatonales}

La aplicación del filtro peatonal basado en rangos de velocidad (0.6 - 2.08 m/s o 2.16 - 7.49 km/h) produjo los siguientes resultados:

\begin{table}[h]
	\centering
	\begin{tabular}{@{}ccc@{}}
		\toprule
		\textbf{Métrica}                                                                      & \textbf{Valor} & \textbf{Interpretación}                                                                              \\ \midrule
		\begin{tabular}[c]{@{}c@{}}Puntos peatonales\\ válidos\end{tabular}                   & 171,461        & \begin{tabular}[c]{@{}c@{}}Total de puntos GPS dentro del rango\\ de velocidad peatonal\end{tabular} \\
		\begin{tabular}[c]{@{}c@{}}Usuarios únicos con\\ trayectorias peatonales\end{tabular} & 41,882         & 32.5\% de los individuos clasificados                                                                \\
		\begin{tabular}[c]{@{}c@{}}Periodo temporal\\ cubierto\end{tabular}                   & 6-14 Nov 2022  & 9 días de actividad monitoreada                                                                      \\
		Velocidad promedio                                                                    & 1.17 m/s       & Consistente con caminata humana normal                                                               \\
		\begin{tabular}[c]{@{}c@{}}Velocidad máxima\\ registrada\end{tabular}                 & 2.08 m/s       & Límite superior del filtro aplicado                                                                  \\ \bottomrule
	\end{tabular}
	\caption{Resultados de la extracción de trayectorias peatonales}
	\label{tab:resultados_peatonales}
\end{table}

\paragraph{Análisis de reducción por filtrado}
\begin{itemize}
	\item \textbf{Reducción significativa}: De 129,920 individuos clasificados se redujo a 41,882 con trayectorias peatonales válidas, lo que representa una reducción del 67.8\%.
	\item \textbf{Interpretación}: Aproximadamente dos tercios de los individuos clasificados usaban predominantemente transporte motorizado o tenían patrones de movimiento fuera del rango peatonal.
	\item \textbf{Calidad del dataset resultante}: Los 41,882 individuos restantes representan una muestra de alta calidad para análisis de movilidad peatonal específica.
\end{itemize}

\section{Resultados del Análisis de Clustering (K-Means)}

\subsection{Configuración y distribución general}

Se aplicó el algoritmo K-Means con $k=4$ clusters sobre 14 características extraídas de las trayectorias de calidad:

\begin{table}[h]
	\centering
	\begin{tabular}{|l|c|c|p{8cm}|}
		\hline
		\textbf{Cluster} & \textbf{Cantidad} & \textbf{Porcentaje} & \textbf{Denominación} \\
		\hline
		Cluster 0 & $\sim$26,000 & 26\% & Bajo volumen \\
		\hline
		Cluster 1 & $\sim$25,000 & 25\% & Super usuarios \\
		\hline
		Cluster 2 & $\sim$24,000 & 24\% & Usuarios moderados \\
		\hline
		Cluster 3 & $\sim$25,000 & 25\% & Usuarios irregulares \\
		\hline
	\end{tabular}
	\caption{Distribución de individuos por cluster (valores aproximados)}
	\label{tab:distribucion_clusters}
\end{table}

\paragraph{Calidad del clustering}
\begin{itemize}
	\item \textbf{Varianza explicada por PCA}: Las dos primeras componentes principales explican el 47.62\% de la varianza total:
	\begin{itemize}
		\item PC1: 27.91\% (mayor variabilidad)
		\item PC2: 19.71\% (segunda mayor variabilidad)
	\end{itemize}
	\item \textbf{Separación visual}: Los clusters muestran separación clara en el espacio PCA, indicando agrupaciones bien definidas y diferenciadas.
\end{itemize}

\subsection{Caracterización detallada de clusters}

\begin{table}[h]
	\centering
	\begin{tabular}{|p{2.5cm}|c|c|c|c|c|}
		\hline
		\textbf{Cluster / Métrica} & \textbf{Quality Score} & \textbf{Movement Points} & \textbf{Spatial Range} & \textbf{Active Days} & \textbf{Records Count} \\
		\hline
		\textbf{Cluster 1: Super Usuarios} & 66.32 & 35.82 & 0.56 & 5.74 & 486.79 \\
		\hline
		\textbf{Cluster 2: Usuarios Moderados} & 49.65 & 8.69 & 0.47 & 5.75 & - \\
		\hline
		\textbf{Cluster 0: Bajo Volumen} & 43.51 & 4.21 & 0.42 & 3.36 & - \\
		\hline
		\textbf{Cluster 3: Usuarios Irregulares} & 45.82 & 3.33 & 0.21 & 2.94 & - \\
		\hline
	\end{tabular}
	\caption{Caracterización cuantitativa de los clusters identificados}
	\label{tab:caracterizacion_clusters}
\end{table}

\paragraph{Perfiles comportamentales}
\begin{enumerate}
	\item \textbf{Cluster 1: "Super Usuarios"} (25\%):
	\begin{itemize}
		\item \textbf{Perfil}: Usuarios con patrones de movilidad intensos y excelente calidad de datos.
		\item \textbf{Interpretación}: Representan el segmento ideal para modelado de trayectorias, con suficiente información temporal y espacial para capturar patrones complejos de movilidad urbana.
	\end{itemize}
	
	\item \textbf{Cluster 2: "Usuarios Moderados"} (24\%):
	\begin{itemize}
		\item \textbf{Característica distintiva}: Precisión GPS promedio de 21.28 metros (la más baja de todos los clusters).
		\item \textbf{Perfil}: Usuarios regulares con actividad sostenida pero menor precisión GPS, sugiriendo uso de A-GPS o triangulación WiFi.
		\item \textbf{Aplicación}: Útiles para análisis de patrones generales pero con menor resolución espacial.
	\end{itemize}
	
	\item \textbf{Cluster 0: "Bajo Volumen"} (26\%):
	\begin{itemize}
		\item \textbf{Perfil}: Usuarios con baja actividad y movilidad reducida.
		\item \textbf{Interpretación}: Probablemente usuarios ocasionales o personas con rutinas muy localizadas.
		\item \textbf{Aplicación}: Menos útiles para modelado general pero valiosos para estudiar patrones de movilidad local.
	\end{itemize}
	
	\item \textbf{Cluster 3: "Usuarios Irregulares"} (25\%):
	\begin{itemize}
		\item \textbf{Características distintivas}: Time Span de solo 2.32 días (período de observación muy corto).
		\item \textbf{Perfil}: Usuarios con patrones muy irregulares o períodos de monitoreo breves.
		\item \textbf{Interpretación}: Probablemente representan uso esporádico de la aplicación o instalaciones/desinstalaciones frecuentes.
	\end{itemize}
\end{enumerate}

\subsection{Análisis de correlaciones entre características}

La matriz de correlación reveló relaciones significativas que validan el diseño del algoritmo de puntuación:

\begin{table}[h]
	\centering
	\begin{tabular}{|p{6cm}|c|p{8cm}|}
		\hline
		\textbf{Relación} & \textbf{Coeficiente (r)} & \textbf{Interpretación} \\
		\hline
		\texttt{quality\_score} $\leftrightarrow$ \texttt{records\_count} & 0.67 & Más datos implica mejor calidad (correlación positiva fuerte) \\
		\hline
		\texttt{quality\_score} $\leftrightarrow$ \texttt{score\_volume} & 0.75 & El volumen impacta directamente la calidad \\
		\hline
		\texttt{movement\_points} $\leftrightarrow$ \texttt{score\_mobility} & 0.82 & Coherencia entre métricas de movilidad \\
		\hline
		\texttt{time\_span\_days} $\leftrightarrow$ \texttt{score\_duration} & 1.00 & Relación perfecta por diseño del algoritmo \\
		\hline
		\texttt{score\_accuracy} $\leftrightarrow$ \texttt{avg\_accuracy\_meters} & -1.00 & Relación inversa lógica: menor error = mayor score \\
		\hline
	\end{tabular}
	\caption{Correlaciones significativas identificadas}
	\label{tab:correlaciones_clustering}
\end{table}

\paragraph{Implicación}: Las características seleccionadas capturan aspectos complementarios de la calidad de las trayectorias, validando el diseño del algoritmo de puntuación compuesta.

\subsection{Distribuciones de características clave}

\begin{enumerate}
	\item \textbf{Quality Score}:
	\begin{itemize}
		\item Distribución sesgada hacia valores altos (concentración en 60-80 puntos).
		\item Refleja el filtrado previo por calidad mínima ($\geq$35 puntos).
	\end{itemize}
	
	\item \textbf{Movement Points}:
	\begin{itemize}
		\item Gran concentración en valores bajos ($<$100 puntos).
		\item Cola larga hacia valores altos (hasta 500+ puntos).
		\item Distribución típica de datos de movilidad con pocos usuarios muy activos.
	\end{itemize}
	
	\item \textbf{Spatial Range}:
	\begin{itemize}
		\item Mayoría con rango espacial pequeño ($<$2.0 grados, $\sim$220 km).
		\item Indica que la mayoría de usuarios se mueve dentro de áreas urbanas locales.
	\end{itemize}
	
	\item \textbf{Active Days}:
	\begin{itemize}
		\item Distribución bimodal con picos en:
		\begin{itemize}
			\item 3-4 días (usuarios de fin de semana o muy ocasionales).
			\item 8-9 días (usuarios con cobertura casi completa del período estudiado).
		\end{itemize}
	\end{itemize}
\end{enumerate}

\section{Resultados del Análisis de Longitudes de Vuelo}

El análisis de distribución de longitudes de trayectoria reveló características extremas que evidencian patrones de movilidad muy heterogéneos.

\subsection{Estadísticas descriptivas básicas}

\begin{table}[h]
	\centering
	\begin{tabular}{|p{4cm}|c|p{8cm}|}
		\hline
		\textbf{Métrica} & \textbf{Valor} & \textbf{Interpretación} \\
		\hline
		Total de individuos & 41,882 & Muestra analizada \\
		\hline
		Media & 3.48 km & Longitud promedio de trayectoria \\
		\hline
		Desviación estándar & 28.14 km & Variabilidad extremadamente alta \\
		\hline
		Mínimo & 0.0 km & Usuarios sin desplazamiento detectable \\
		\hline
		Q1 (25\%) & 0.0 km & Un cuarto de usuarios no se movió \\
		\hline
		Mediana (50\%) & 0.0 km & La mitad de usuarios tiene longitud cero \\
		\hline
		Q3 (75\%) & 0.63 km & Tercer cuartil apenas 630 metros \\
		\hline
		Máximo & 1,746.08 km & Outlier extremo (Tijuana a CDMX) \\
		\hline
	\end{tabular}
	\caption{Estadísticas descriptivas de longitudes de vuelo}
	\label{tab:estadisticas_longitudes}
\end{table}

\subsection{Métricas de forma de la distribución}

\begin{table}[h]
	\centering
	\begin{tabular}{|p{4cm}|c|p{8cm}|}
		\hline
		\textbf{Métrica} & \textbf{Valor} & \textbf{Interpretación} \\
		\hline
		Asimetría (Skewness) & 31.06 & Valor extremadamente alto ($>$3 se considera muy sesgado) \\
		\hline
		Curtosis (Kurtosis) & 1,341.15 & Valor excepcional ($>$3 indica colas pesadas) \\
		\hline
		Coeficiente de Variación (CV) & 8.08 & Variabilidad extrema relativa a la media \\
		\hline
	\end{tabular}
	\caption{Métricas de forma de la distribución de longitudes}
	\label{tab:metricas_forma}
\end{table}

\paragraph{Interpretación de las métricas de forma}:
\begin{itemize}
	\item \textbf{Skewness = 31.06}: Indica cola derecha extraordinariamente larga, con la mayoría de datos concentrados en valores bajos con outliers masivos.
	\item \textbf{Kurtosis = 1,341.15}: Sugiere distribución leptocúrtica extrema: pico muy pronunciado en cero y colas muy pesadas.
	\item \textbf{CV = 8.08}: Confirma heterogeneidad extrema en los patrones de movilidad; la desviación estándar es 8 veces mayor que la media.
\end{itemize}

\subsection{Distribución inflada en cero}

El hallazgo más significativo es la distribución inflada en cero con tres segmentos claramente diferenciados:

\begin{table}[h]
	\centering
	\begin{tabular}{|p{4cm}|c|c|p{7cm}|}
		\hline
		\textbf{Segmento} & \textbf{Rango} & \textbf{Cantidad} & \textbf{Interpretación y posibles causas} \\
		\hline
		Usuarios Sedentarios & 0.0 km & 20,941 (50\%) & \begin{itemize}[leftmargin=*]
			\item Usuarios que permanecieron en una única ubicación
			\item Dispositivos estáticos (olvidados en casa/oficina)
			\item Datos de referencia sin movimiento
			\item Errores en estimación para trayectorias muy cortas
		\end{itemize} \\
		\hline
		Usuarios Locales & 0.01 - 0.63 km & 10,470 (25\%) & \begin{itemize}[leftmargin=*]
			\item Desplazamientos dentro del vecindario
			\item Compras en tiendas cercanas
			\item Paseos cortos recreativos
			\item Movilidad intra-edificio
		\end{itemize} \\
		\hline
		Usuarios Móviles & $>$0.63 km & 10,471 (25\%) & \begin{itemize}[leftmargin=*]
			\item Commuters (casa-trabajo)
			\item Usuarios de transporte público
			\item Deportistas (corredores, ciclistas)
			\item Viajeros ocasionales
		\end{itemize} \\
		\hline
	\end{tabular}
	\caption{Segmentación de usuarios por patrón de movilidad}
	\label{tab:segmentacion_movilidad}
\end{table}

\subsection{Ajuste de distribuciones teóricas}

Se intentó ajustar seis distribuciones probabilísticas comunes a los datos:

\begin{table}[h]
	\centering
	\begin{tabular}{|l|c|c|c|p{6cm}|}
		\hline
		\textbf{Distribución} & \textbf{KS Statistic} & \textbf{p-value} & \textbf{AIC} & \textbf{Interpretación} \\
		\hline
		Rayleigh & 0.4550 & 0.0 & 368,039.06 & Ajuste muy pobre \\
		\hline
		Gamma & 0.7378 & 0.0 & -1,978,645.13 & Ajuste extremadamente pobre \\
		\hline
		Lognormal & - & - & - & No convergió \\
		\hline
		Normal & - & - & - & No aplicable (datos no negativos) \\
		\hline
		Exponencial & - & - & - & No ajustó \\
		\hline
		Weibull & - & - & - & No ajustó \\
		\hline
	\end{tabular}
	\caption{Resultados del ajuste de distribuciones teóricas}
	\label{tab:ajuste_distribuciones}
\end{table}

\paragraph{Conclusiones del ajuste}:
\begin{itemize}
	\item \textbf{Ninguna distribución estándar ajusta adecuadamente} los datos debido a:
	\begin{itemize}
		\item Inflación de ceros (50\% de observaciones en cero).
		\item Heterogeneidad extrema entre segmentos.
	\end{itemize}
	\item \textbf{KS Statistics $>$0.1} indican mal ajuste (valores obtenidos $>>$0.4).
	\item \textbf{p-values = 0.0} rechazan todas las hipótesis de ajuste.
\end{itemize}

\paragraph{Implicación para modelado}: Se requiere un \textbf{modelo de mezcla (mixture model)} que combine:
\begin{itemize}
	\item Componente discreta para la masa de probabilidad en cero (50\%).
	\item Distribución continua (posiblemente lognormal o Weibull truncada) para valores $>$0.
\end{itemize}

\subsection{Análisis de outliers extremos}

\paragraph{Outlier máximo}: 1,746.08 km (equivalente a la distancia entre Tijuana y Ciudad de México).

\begin{itemize}
	\item \textbf{Possibles causas}:
	\begin{itemize}
		\item Errores de GPS (saltos de ubicación por mala señal).
		\item Inclusión de segmentos de viaje no peatonal a pesar del filtrado.
		\item Múltiples días de actividad sumados incorrectamente.
		\item Trayectorias de viajes largos (vacaciones, viajes de negocios).
	\end{itemize}
	
	\item \textbf{Impacto en el análisis}:
	\begin{itemize}
		\item Distorsiona la media hacia arriba (3.48 km no es representativa del usuario típico).
		\item La mediana (0.0 km) es más robusta pero oculta información de usuarios móviles.
		\item Justifica el uso de transformación logarítmica para visualizaciones.
	\end{itemize}
\end{itemize}

\paragraph{Recomendaciones para modelado futuro}:
\begin{enumerate}
	\item Limitar longitudes máximas (ej., usando percentil 99).
	\item Analizar separadamente los tres segmentos identificados (sedentarios, locales, móviles).
	\item Usar medianas en lugar de medias para reportar tendencias centrales.
\end{enumerate}

\section{Resultados del Análisis de Tiempos de Pausa}

El análisis de pausas reveló patrones de comportamiento estacionario que complementan la caracterización de movimiento.

\subsection{Cobertura y estadísticas generales}

\begin{table}[h]
	\centering
	\begin{tabular}{|p{5cm}|c|p{7cm}|}
		\hline
		\textbf{Métrica} & \textbf{Valor} & \textbf{Interpretación} \\
		\hline
		Total de pausas detectadas & 1,134 & Eventos de pausa identificados \\
		\hline
		Usuarios con pausas detectables & 739 & Solo 1.8\% del total de usuarios \\
		\hline
		Sesiones únicas con pausas & 867 & Períodos de actividad analizados \\
		\hline
		Ratio pausas/sesión & 1.3 & Promedio de pausas por período activo \\
		\hline
	\end{tabular}
	\caption{Cobertura del análisis de tiempos de pausa}
	\label{tab:cobertura_pausas}
\end{table}

\paragraph{Observación crítica}: Solo el 1.8\% de usuarios (739 de 41,882) mostraron pausas detectables, sugiriendo:
\begin{itemize}
	\item Los umbrales de detección (distancia $<$50m, tiempo $>$1 segundo) son muy restrictivos.
	\item La mayoría de trayectorias son demasiado cortas para contener pausas significativas.
	\item Muchos usuarios tienen solo unos pocos puntos GPS, insuficientes para detectar pausas.
\end{itemize}

\subsection{Distribución de duraciones de pausa}

\begin{table}[h]
	\centering
	\begin{tabular}{|p{4cm}|c|p{8cm}|}
		\hline
		\textbf{Estadística} & \textbf{Valor} & \textbf{Interpretación} \\
		\hline
		Media & 32.0 minutos & \textit{Promedio} de duración de pausas \\
		\hline
		Mediana & 14.0 minutos & \textit{Valor típico} de duración de pausas \\
		\hline
		Desviación estándar & 40.1 minutos & Variabilidad extremadamente alta \\
		\hline
		Mínimo & 0.0167 min (1s) & Pausa más corta detectada \\
		\hline
		Máximo & 472.3 min (7.9h) & Pausa más larga detectada \\
		\hline
	\end{tabular}
	\caption{Estadísticas de duración de pausas}
	\label{tab:estadisticas_pausas}
\end{table}

\paragraph{Análisis de la dispersión}:
\begin{itemize}
	\item \textbf{Media (32 min) $>>$ Mediana (14 min)}: Distribución fuertemente sesgada a la derecha.
	\item \textbf{Desviación estándar (40.1 min) $>$ Media (32 min)}: Variabilidad extremadamente alta.
	\item \textbf{Interpretación}: La mayoría de pausas son cortas, pero algunas son muy largas (colas pesadas).
\end{itemize}

\subsection{Segmentación por duración}

\begin{table}[h]
	\centering
	\begin{tabular}{|p{4cm}|c|c|c|}
		\hline
		\textbf{Categoría} & \textbf{Duración} & \textbf{Cantidad} & \textbf{Porcentaje} \\
		\hline
		Pausas muy cortas & $<$5 min & 367 & 32.4\% \\
		\hline
		Pausas cortas & 5-15 min & 215 & 19.0\% \\
		\hline
		Pausas medias & 15-30 min & 145 & 12.8\% \\
		\hline
		Pausas largas & $>$30 min & 407 & 35.9\% \\
		\hline
	\end{tabular}
	\caption{Distribución de pausas por categoría de duración}
	\label{tab:distribucion_pausas}
\end{table}

\paragraph{Patrón bimodal identificado}:
\begin{enumerate}
	\item \textbf{Pico 1}: Pausas muy cortas (32.4\%) - Componente de actividades cotidianas rápidas.
	\item \textbf{Valle}: Pausas medias (12.8\%) - Zona de transición poco común.
	\item \textbf{Pico 2}: Pausas largas (35.9\%) - Componente de estancias prolongadas.
\end{enumerate}

\paragraph{Interpretación comportamental}:
\begin{itemize}
	\item \textbf{Pausas muy cortas} ($<$5 min): Semáforos, esperas breves, paradas de tránsito.
	\item \textbf{Pausas cortas} (5-15 min): Compras rápidas, encuentros sociales, café rápido.
	\item \textbf{Pausas medias} (15-30 min): Comidas ligeras, reuniones breves, descansos.
	\item \textbf{Pausas largas} ($>$30 min): Trabajo, hogar, comidas completas, actividades programadas.
\end{itemize}

\paragraph{Implicación fundamental}: Las personas tienden a hacer pausas muy cortas (pocos minutos) o muy largas (más de media hora), pero raramente pausas de duración intermedia. Esto refleja la naturaleza discreta de las actividades humanas: se está "en movimiento" o "detenido por tiempo extendido" para realizar actividades significativas.

\endinput