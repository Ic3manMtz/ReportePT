\section{Importancia de los modelos de movilidad para la evaluación de protocolos para redes móviles}
La simulación de una red de comunicaciones en donde intervienen dispositivos
personales de comunicaciones (i.e., teléfonos celulares) requiere contar con
modelos que representen fielmente los patrones de movimiento de las personas. De
otra manera, la utilidad de las conclusiones que se puedan obtener de esa
simulación es limitada. Aunque ya existen muchas propuestas de modelos para
proceder con esas simulaciones, la mayoría contemplan que los individuos nos
movemos solos y, sin embargo, sabemos que esto no siempre es así ya que, en
ocasiones, nos movemos en grupos.
Para avanzar hacia la definición de un modelo de movilidad humana, en este
proyecto, se propone la caracterización de datos de trayectorias peatonales individuales. Esto
nos permitirá obtener trayectorias con datos obtenidos de dispositivos personales de comunicaciones.

\section{Proceso de diseño de un modelo de movilidad}



\section{Objetivo del proyecto}


\section{Logros}


\section{Estructura del documento}
